\documentclass {article}

\usepackage{amssymb}
\newcommand\conf{\mathbf{q}}
\newcommand\confoffset{\mathbf{q}_{off}}
\newcommand\transf{T}
\newcommand\se[1]{\mathfrak{se}(#1)}
\newcommand\transl{\mathbf{t}}
\newcommand\linvel{\mathbf{v}}
\newcommand\angvel{\omega}
\newcommand\cross[1]{\left[#1\right]_{\times}}
\newcommand\x{\mathbf{x}}
\newcommand\y{\mathbf{y}}

\title {Calibration of Pyrene arms and camera}
\author {Florent Lamiraux}
\date {}
\begin {document}
\maketitle
\section {Introduction}

The objective of this note is to process data collected with Pyrene in order
to calibrate the sub-kinematic chain of the arms and the head, including the
position of Apriltags supports on the wrists and the position of the camera on the head.

\section {Position of the problem}

The data we try to evaluate are the following
\begin{itemize}
\item $\confoffset$ a vector of differences between the angular values of each joint ($\conf$) and the measured value ($\hat{\conf}$): $\conf = \hat{\conf} + \confoffset$,
\item $^{h}\transf_{c}$ the position of the camera in the head (joint \texttt{talos/head\_2\_joint})
\item $^{lw}\transf_{ls}$ the position of the left tag support in the left wrist,
\item $^{rw}\transf_{rs}$ the position of the right tag support in the right
  wrist.
\end{itemize}
We denote by $^{w}\transf_{s}$ to refer indifferently to one of two above variables.
Note that only components of $\confoffset$ corresponding to joints between the wrists and the head can be evaluated.

\subsection {Input data}

The input data consists of a list of $n$ measures of the form:
\begin{itemize}
  \item $m_i = (\hat{\conf}_i, ^{c}\hat\transf_{s\ i})$, $1 \leq i \leq n$,
\end{itemize}
where $s$ stands for left or right support depending on the wrist that has been detected.

\subsection {Error functions}

For each measure, we define an error function as follows:
\begin{equation}\label{eq:constraint}
\log_{SE(3)}\left(^{c}\hat{\transf}_{s,i}^{-1}\ ^{c}{\transf}_{s} (\hat{\conf}_{i} + \confoffset)\right)
\end{equation}
that represents the position of the support in the camera frame as
computed by the forward kinematics using the model of the robot.
$$
^{c}{\transf}_{s} (\hat{\conf}_{i} + \confoffset) =\ ^{h}\transf_{c}^{-1}\ \transf_{h}^{-1}(\hat{\conf}_{i} + \confoffset)\ \transf_{w} (\hat{\conf}_{i} + \confoffset)\ ^{w}\transf_{s}
$$
Using expressions~(\ref{eq:derivative-inverse}) and (\ref{eq:derivative-product}), and omitting function arguments, the derivative of this expression is
\begin{eqnarray*}
^{c}{\dot{\transf}}_{s} &=& -^{w}X_{s}^{-1}\ X_{w}^{-1}\ X_{h}\ ^{h}X_{c}\ ^{h}\nu_{c}\ -\ 
^{w}X_{s}^{-1}\ X_{w}^{-1}\ X_{h}\ \nu_{h}\ +\ ^{w}X_{s}^{-1}\ \nu_{w}\ +\ ^{w}\nu_{s}\\
&=& -^{s}X_{c}\ ^{h}\nu_{c}\ -\ 
^{s}X_{h}\ \nu_{h}\ +\ ^{s}X_{w}\ \nu_{w}\ +\ ^{w}\nu_{s}\end{eqnarray*}

\subsection{Variables}

The variables of expression (\ref{eq:constraint}) are
$$
\x = \left(\confoffset,\ ^{h}\transf_{c},\ ^{lw}\transf_{ls},\ ^{lw}\transf_{ls}\right).
$$

\subsection{Jacobian}

The Jacobian of each constraint can be written as:
\begin{equation}\label{eq:error-jacobian}
J = J_{logSE(3)}
\left(\begin{array}{cccc}
  -^{s}X_{h}\ J_{h} %(\hat{\conf}_{i} + \confoffset)
  +\ ^{s}X_{w}\ J_{w}%(\hat{\conf}_{i} + \confoffset)
  & -^{s}X_{c}
  & J_{s}
\end{array}
\right)
\end{equation}
where
$$
J_{s} =
\left(\begin{array}{cc}
I_6 & 0
\end{array}\right)
$$
if the left support is measured,
$$
J_{s} =
\left(\begin{array}{cc}
  0 & I_6
\end{array}\right)
$$
if the right support is measured.

\section {Resolution}

We wish to minimize the sum of the error functions:
$$
C(\x) = \frac{1}{2} \sum_{i} \|\log_{SE(3)}\left(^{c}\hat{\transf}_{s,i}^{-1}\ ^{c}{\transf}_{s} (\hat{\conf}_{i} + \confoffset)\right)\|^2
$$
To do so, we stack the error functions in a vector
$$
\y = \left(\begin{array}{c}
  \log_{SE(3)}\left(^{c}\hat{\transf}_{s,1}^{-1}\ ^{c}{\transf}_{s} (\hat{\conf}_{1} + \confoffset)\right)\\
  \vdots\\
  \log_{SE(3)}\left(^{c}\hat{\transf}_{s,n}^{-1}\ ^{c}{\transf}_{s} (\hat{\conf}_{n} + \confoffset)\right)
\end{array}\right)
$$
With this definition, we can write
$$
C(\x) = \frac{1}{2}\|\y\|^2
$$
and
$$
\nabla{C} = \y^T\frac{\partial\y}{\partial\x}
$$
where $\frac{\partial\y}{\partial\x}$ is the matrix obtained by stacking
Jacobian matrices~(\ref{eq:error-jacobian}).

\section {Appendix: motions on SE(3)}

Let $\transf$ be a mapping from time to $SE(3)$, and let
$$M=\left(\begin{array}{cc}R & \transl \\ 0 & 1\end{array}\right)$$
be the representation of $\transf$ as an homogeneous matrix.

We denote by
\begin{equation}\label{eq:screw}
  \dot{\transf}=\nu=\left(\begin{array}{c}\linvel\\\angvel\end{array}\right)\in \se{3}
\end{equation}
the derivative of $\transf$. Then
$$
\dot{M} = \left(\begin{array}{cc}R\cross{\angvel} & R\linvel\\
  0 & 0\end{array}\right)
$$
\begin{itemize}
\item $R\linvel$ is the linear velocity of the image by $\transf$ of the origin, expressed in global coordinate frame,
\item $\linvel$ is the same velocity expressed in  $\transf$ coordinate frame,
\item $R$ is the rotation matrix relative to $\transf$,
\item $R\angvel$ is the angular velocity vector of $\transf$ expressed in
  global coordinate frame,
\item $\angvel$ is the same vector expressed in $\transf$ coordinate frame,
\item $\cross{\angvel}$ is the antisymmetric matrix relative to vector $\angvel$.
\end{itemize}

We denote
$$
X = \left(\begin{array}{cc}R&\cross{\transl}R\\0&R\end{array}\right)
$$
Then,
\begin{eqnarray}
  \label{eq:derivative-inverse}
  \frac{d}{dt}(\transf^{-1}) &=& -X\nu,\\
  \frac{d}{dt}(\transf_1.\transf_2) &=&X_2^{-1}\nu_1+\nu_2,
\end{eqnarray}
and more generally
\begin{equation}\label{eq:derivative-product}
\frac{d}{dt}(\transf_1\cdots\transf_n) = X_n^{-1}\cdots X_2^{-1}\nu_1 + X_n^{-1}\cdots X_3^{-1}\nu_2 + \cdots \nu_n
\end{equation}
\end {document}
